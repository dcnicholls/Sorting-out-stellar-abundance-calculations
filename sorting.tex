\section{Background}

The conventions used by stellar astrophysicists to characterise atomic abundances in stellar spectra differ considerably from those used in nebular astrophysics.  In particular, it is normal for stellar abundance values to be reported in terms of their offsets from (standard) solar values. The results depend on the ``standard'' used, and these are numerous. See, for example, \cite{1989GeCoA..53..197A}, \cite{1998SSRv...85..161G}, \cite{2007SSRv..130..105G}, \cite{2009ARA&A..47..481A}, \cite{2010Ap&SS.328..179G}, \cite{2015A&A...573A..25S}, \cite{2015A&A...573A..26S} and \cite{2015A&A...573A..27G}.  If the solar standard used is not specified, or if normalisation is done in a non-standard way, it may be impossible convert the observations to a different standard. Where simple scaling to a specified set of solar abundances is used, it is still a potentially error-prone exercise to convert the data to a different standard. This note derives the formulae that allow conversion between different solar standards. We also provide a short appendix on the difference between stellar and nebular atomic abundance measurement.
  
\section{Definitions}
The abundance\footnote{Abundance defined in terms of numbers of atoms rather than mass} of oxygen in a stellar spectrum relative to the standard solar abundance is denoted by $[\text{O}]_*$, and the relative or normalised abundance of iron by $[\text{Fe}]_*$, where,
\begin{equation}
[\text{O}]_* = \log_{10}\left(\frac{\text{O}}{\text{H}}\right)_* - \log_{10}\left(\frac{\text{O}}{\text{H}}\right)_\odot
\end{equation}
and
\begin{equation}
[\text{Fe}]_* = \log_{10}\left(\frac{\text{Fe}}{\text{H}}\right)_* - \log_{10}\left(\frac{\text{Fe}}{\text{H}}\right)_\odot .
\end{equation}
Here, the subscript ``$_\odot$'' refers to solar abundances and ``$_*$'' to stellar abundances.

\section{Discussion}
Let us denote solar abundance standard 1 with the subscript $_1$ and standard 2 with $_2$. We want to express the normalised abundances in one standard in terms of the normalised abundances in a second standard. Let
\begin{equation}
x_1 = [\text{O}]_{*1}\ ,\  x_2 = [\text{O}]_{*2}\ ,\  y_1 = [\text{Fe}]_{*1}\ ,\  y_2 = [\text{Fe}]_{*2}\ .
\end{equation}
Here, $x_1$ and $x_2$ are the normalised oxygen abundance values for a star in the two abundance standards, and similarly, $y_1$ and $y_2$ for iron.
Let us also put
\begin{equation}
a_* = \log_{10}\left(\frac{\text{O}}{\text{H}}\right)_* ,\  b_* =  \log_{10}\left(\frac{\text{Fe}}{\text{H}}\right)_*\ .
\end{equation}
and
\begin{equation}
c_1 = \log_{10}\left(\frac{\text{O}}{\text{H}}\right)_{\odot 1} ,\  c_2 =  \log_{10}\left(\frac{\text{O}}{\text{H}}\right)_{\odot 2} ,\  d_1 = \log_{10}\left(\frac{\text{Fe}}{\text{H}}\right)_{\odot 1} ,\  d_2 =  \log_{10}\left(\frac{\text{Fe}}{\text{H}}\right)_{\odot 2}\ .
\end{equation}
Here, $a_*$ and $b_*$ are the measured (absolute) abundances of oxygen and iron for the star. $c_1$ and $c_2$ are the solar oxygen abundances in standards 1 and 2 respectively, and $d_1$ and $d_2$ are the the solar iron abundances in standards 1 and 2.
  
From equations 1 through 5, we have
\begin{equation}
x_1 = a_* - c_1 ,\  x_2 = a_* - c_2 ,\ y_1 = b_* - d_1 ,\ y_2 = b_* - d_2 \ .
\end{equation}

Eliminating $a_*$ and $b_*$, we may use Equation 6 to derive expressions for the normalised abundances ($x_2, y_2$) in standard 2 in terms of the normalised abundances ($x_1, y_1$)  in standard 1, as follows,
\begin{equation}
x_2 = x_1 + c_1 - c_2 
\end{equation}
and
\begin{equation}
y_2 = y_1 + d_1 - d_2 \ .
\end{equation}

We may also wish to convert elemental abundances relative to iron (e.g., [O/Fe]) between the two standards. Since these are logarthimic values,
\begin{equation}
[\text{O}/\text{Fe}] = [\text{O}] - [\text{Fe}] \ .
\end{equation}
and, in terms of the parameters above,
\begin{equation}
[\text{O}/\text{Fe}]_{*1} = x_1 - y_1 \ ,\  [\text{O}/\text{Fe}]_{*2} = x_2 - y_2 \ .
\end{equation}
It follows from Equations 7, 8 and 10 that
\begin{equation}
[\text{O}/\text{Fe}]_{*2} = [\text{O}/\text{Fe}]_{*1} + c_1 - c_2 - d_1 +d_2 \ .
\end{equation}

Expressed in physically meaningful terms,
\begin{equation}
[\text{O}/\text{Fe}]_{*2} = [\text{O}/\text{Fe}]_{*1} + \log_{10}\left(\frac{\text{O}}{\text{H}}\right)_{\odot 1} - \log_{10}\left(\frac{\text{O}}{\text{H}}\right)_{\odot 2} - \log_{10}\left(\frac{\text{Fe}}{\text{H}}\right)_{\odot 1} + \log_{10}\left(\frac{\text{Fe}}{\text{H}}\right)_{\odot 2} \ .
\end{equation}

Also,
\begin{equation}
[\text{Fe}]_{*2} = [\text{Fe}]_{*1} + \log_{10}\left(\frac{\text{Fe}}{\text{H}}\right)_{\odot 1} - \log_{10}\left(\frac{\text{Fe}}{\text{H}}\right)_{\odot 2} \ .
\end{equation}

Given that we know the two sets of solar standards, Equations 12 and 13 allow us to convert normalised abundances and abundance ratios between the two standards.

\section{Example}
Let us assume we wish to convert the normalised iron abundance [Fe]$_*$ and the normalised oxygen to iron ratio [O/Fe]$_*$ from standards used by the observer in reporting their results to the local galactic concordance (LGC) standard used by the photoionisation modelling code, Mappings V.  In the LGC, in the A$_\text{X}$ notation\footnote{ A$_\text{X}$ = 12 + $\log_{10}(\frac{\text{X}}{\text{H}})$, a form devised in the early days of computing to avoid the need for minus signs.}, A$_\text{O}$ = 12+$\log_{10}(\text{O/H})_\odot$ = 8.76 and A$_\text{Fe}$ = 7.52 .  (Note that we can use either $\log_{10}(\text{O/H})$ or 12+$\log_{10}(\text{O/H})$ as the factors of 12 cancel.)

Consider observations reported using the \cite{2009ARA&A..47..481A} solar standards, where A$_\text{O}$ = 8.69 and A$_\text{Fe}$ = 7.50 . From Equations 12 and 13 we have
\begin{equation}
[\text{O}/\text{Fe}]_{*\text{LGC}} = [\text{O}/\text{Fe}]_{*\text{original}} + 8.69 - 8.76 - 7.50 + 7.52 = [\text{O}/\text{Fe}]_{*\text{original}} -0.05 \ .
\end{equation}
and
\begin{equation}
[\text{Fe}]_{*\text{LGC}} = [\text{Fe}]_{*\text{original}} + 7.50 - 7.52 = [\text{Fe}]_{*\text{original}} - 0.02 \ .
\end{equation}

\section{Comment}
The conversion process is simple, but the difficult part is usually identifying the standards used by observers to normalise their abundance values.

\section{Appendix}
\subsection{Note on differences between nebular and stellar abundance measurement}
Finally, it is useful to mention the different approaches taken in stellar and nebular abundance measurement. Absorption lines from iron (usually neutral and singly ionised atoms) are strong---or, at least, present---in virtually all stellar spectra and can be measured accurately, enabling reliable iron abundances to be estimated. Thus, iron abundance as a fraction of hydrogen (log$_{10}$(Fe/H)) is used as the standard abundance scale for stellar atmospheres. Oxygen, in contrast, is difficult to measure in stellar spectra. It is measurable only in some stellar atmospheres, using the neutral oxygen lines (particularly the 6300\AA\ line) and OH lines in the ultraviolet.  In stellar astrophysics, the convention is to report atomic abundances as the logarithmic ratios with hydrogen [X] (log$_{10}$(X/H)) or as fractions of the iron abundance, [X/Fe], with [Fe] providing the scale for total metallicity.

In ultraviolet, visible and infrared nebular spectra, oxygen emission lines are among the strongest, and are easy to measure accurately. Iron emission lines, conversely, are faint in most nebulae.  Oxygen is the third most common element in the Universe after hydrogen and helium and far more abundant than iron. Consequently oxygen is used as the standard scale for nebular abundance. Total metallicity is usually reported in terms of the oxygen abundance, using the ``12'' notation: e.g., 12+log$_{10}$(O/H), or expressed as a ratio to the solar abundance. Carbon and nitrogen abundances are regularly reports in terms of their ratios to oxygen: (log$_{10}$(C/O) and log$_{10}$(N/O)). 

A complication in nebular abundance measurement is that some species are depleted into dust in varying amounts, while the emission spectra only reflect the gas phase abundances, so it is important to distinguish between gas-phase and total abundances. Estimating dust depletion is problematic. A further significant problem is the standard scale: the standard solar oxygen metallicity has varied by over 0.3 dex over the past few decades (see references), in part because oxygen abundance is difficult to measure in the solar spectrum.  As a result, the "Local Galactic Concordance" (LGC, discussed elsewhere) has been proposed as a solar-independent oxygen metallicity scale. It has the advantage of being based on a statistical average of stellar abundances in the local Milky Way. It can also be measured reliably to much lower metallicities than normally encountered in ionised hydrogen nebulae.

Comparing nebular and stellar abundances to measure total metallicity is a complex process due to the different standard scales.  The two species do not scale consistently.  In the very early Universe, oxygen was generated almost immediately through core-collapse supernovae of large stars, whereas iron generation was delayed until the first type I supernovae. Consequently, at very low metallicities, stellar and nebular abundances, reported in ratios with Fe and O respectively, can given apparently discordant results. 

As low-metallicty stars are far more common than low metallicity nebulae in the local Mlky Way, far more information on the early evolution of relative atomic abundances is available from stellar than from nebular measurements. The formulae in this note can be useful in comparing and converting between standard stellar scales, allowing calibration of low nebular relative abundances from stellar stellar measurements.

  
  
  
  
  
  
  
  
  
  
  
  
  
  
  
  
  
  
  
  
  
  
  
  
  
  
  
  
  
  
  
  
  
  