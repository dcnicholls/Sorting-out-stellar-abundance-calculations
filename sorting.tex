\section{Background}

The conventions used by stellar astrophysicists to characterise atomic abundances in stellar spectra differ considerably from those used in nebular astrophysics.  In particular, it is normal for stellar abundance values to be reported in terms of their offsets from (standard) solar values. The results depend on the ``standard'' used, and these are numerous. See, for example, \cite{1989GeCoA..53..197A}, \cite{1998SSRv...85..161G},\cite{2007SSRv..130..105G}, \cite{2009ARA&A..47..481A}, \cite{2010Ap&SS.328..179G}, \cite{2015A&A...573A..25S}, \cite{2015A&A...573A..26S} and \cite{2015A&A...573A..27G}.  If the solar standard used is not specified, or if normalisation is done in a non-standard way, it may be impossible convert the observations to a different standard. Where simple scaling to a specified set of solar abundances is used, it is still a potentially error-prone exercise to convert the data to a different standard. This note derives the formulae that allow conversion between different solar standards.
  
\section{Definitions}
The abundance of oxygen in a stellar spectrum relative to the standard solar abundance is denoted by $[\text{O}]_*$, and the relative or normalised abundance of iron by $[\text{Fe}]_*$, where,
\begin{equation}
[\text{O}]_* = \log_{10}\left(\frac{\text{O}}{\text{H}}\right)_* - \log_{10}\left(\frac{\text{O}}{\text{H}}\right)_\odot
\end{equation}
and
\begin{equation}
[\text{Fe}]_* = \log_{10}\left(\frac{\text{Fe}}{\text{H}}\right)_* - \log_{10}\left(\frac{\text{Fe}}{\text{H}}\right)_\odot .
\end{equation}
Here, the subscript ``$_\odot$'' refers to solar abundances and ``$_*$'' to stellar abundances.

\section{Discussion}
Let us denote solar abundance standard 1 with the subscript $_1$ and standard 2 with $_2$. We want to express the normalised abundances in one standard in terms of the normalised abundances in a second standard. Let
\begin{equation}
x_1 = [\text{O}]_{*1}\ ,\  x_2 = [\text{O}]_{*2}\ ,\  y_1 = [\text{Fe}]_{*1}\ ,\  y_2 = [\text{Fe}]_{*2}\ .
\end{equation}
Here, $x_1$ and $x_2$ are the normalised oxygen abundance values for a star in the two abundance standards, and similarly, $y_1$ and $y_2$ for iron.
Let us also put
\begin{equation}
a_* = \log_{10}\left(\frac{\text{O}}{\text{H}}\right)_* ,\  b_* =  \log_{10}\left(\frac{\text{Fe}}{\text{H}}\right)_*\ .
\end{equation}
and
\begin{equation}
c_1 = \log_{10}\left(\frac{\text{O}}{\text{H}}\right)_{\odot 1} ,\  c_2 =  \log_{10}\left(\frac{\text{O}}{\text{H}}\right)_{\odot 2} ,\  d_1 = \log_{10}\left(\frac{\text{Fe}}{\text{H}}\right)_{\odot 1} ,\  d_2 =  \log_{10}\left(\frac{\text{Fe}}{\text{H}}\right)_{\odot 2}\ .
\end{equation}
Here, $a_*$ and $b_*$ are the measured (absolute) abundances of oxygen and iron for the star. $c_1$ and $c_2$ are the solar oxygen abundances in standards 1 and 2 respectively, and $d_1$ and $d_2$ are the the solar iron abundances in standards 1 and 2.
  
From equations 1 through 5, we have
\begin{equation}
x_1 = a_* - c_1 ,\  x_2 = a_* - c_2 ,\ y_1 = b_* - d_1 ,\ y_2 = b_* - d_2 \ .
\end{equation}

Eliminting $a_*$ and $b_*$, we may use Equation 6 to derive expressions for the normalised abundances ($x_2, y_2$) in standard 2 in terms of the normalised abundances ($x_1, y_1$)  in standard 1, as follows,
\begin{equation}
x_2 = x_1 + c_1 - c_2 
\end{equation}
and
\begin{equation}
y_2 = y_1 + d_1 - d_2 \ .
\end{equation}

We may also wish to convert elemental abundances relative to iron (e.g., [O/Fe]) between the two standards. Since these are logarthimic values,
\begin{equation}
[\text{O}/\text{Fe}] = [\text{O}] - [\text{Fe}] \ .
\end{equation}
and, in terms of the parameters above,
\begin{equation}
[\text{O}/\text{Fe}]_{*1} = x_1 - y_1 \ ,\  [\text{O}/\text{Fe}]_{*2} = x_2 - y_2 \ .
\end{equation}
It follows from Equations 7, 8 and 10 that
\begin{equation}
[\text{O}/\text{Fe}]_{*2} = [\text{O}/\text{Fe}]_{*1} + c_1 - c_2 - d_1 +d_2 \ .
\end{equation}

Expressed in physically meaningful terms,
\begin{equation}
[\text{O}/\text{Fe}]_{*2} = [\text{O}/\text{Fe}]_{*1} + \log_{10}\left(\frac{\text{O}}{\text{H}}\right)_{\odot 1} - \log_{10}\left(\frac{\text{O}}{\text{H}}\right)_{\odot 2} - \log_{10}\left(\frac{\text{Fe}}{\text{H}}\right)_{\odot 1} + \log_{10}\left(\frac{\text{Fe}}{\text{H}}\right)_{\odot 2} \ .
\end{equation}

Also,
\begin{equation}
[\text{Fe}]_{*2} = [\text{Fe}]_{*1} + \log_{10}\left(\frac{\text{Fe}}{\text{H}}\right)_{\odot 1} - \log_{10}\left(\frac{\text{Fe}}{\text{H}}\right)_{\odot 2} \ .
\end{equation}

Given that we know the two sets of solar standards, Equations 12 and 13 allow us to convert normalised abundances and abundance ratios between the two standards.

\section{Example}
Let us assume we wish to convert the normalised iron abundance [Fe]$_*$ and the normalised oxygen to iron ratio [O/H]$_*$ from standards used by the observer in reporting their results to the local galactic concordance (LGC) standard used by Mappings.  In the LGC, in the A$_\text{X}$ notation \footnote{ A$_\text{X}$ = 12 + $\log_{10}(\frac{\text{X}}{\text{H}})$ }, A$_\text{O}$ = 12+$\log_{10}(\text{O/H})_\odot$ = 8.76 and A$_\text{Fe}$ = 7.52 .

Consider observations reported using the \cite{2009ARA&A..47..481A} solar standards, where A$_\text{O}$ = 8.69 and A$_\text{Fe}$ = 7.50 . From Equations M and N we have
\begin{equation}
[\text{O}/\text{Fe}]_{*2} = [\text{O}/\text{Fe}]_{*1} + 8.69 - 8.76 - 7.50 + 7.52 = [\text{O}/\text{Fe}]_{*1} -0.05 \ .
\end{equation}
and
\begin{equation}
[\text{Fe}]_{*2} = [\text{Fe}]_{*1} + 7.50 - 7.52 = [\text{Fe}]_{*1} - 0.02 \ .
\end{equation}

\section{Comment}
The conversion process is simple, but the difficult part is usually identifying the standards used by observers to normalise their abundance values.

  
  
  
  
  
  
  
  
  
  