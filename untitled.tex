\section{Purpose}

The conventions used by stellar astrophysicists to characterise atomic abundances in stellar spectra appear idiosyncratic to a nebular astrophysicist.  In particular, the normalisation of abundance values to standard solar values depends on the ``standard'' used, and these are numerous. See, for example, \cite{Anders_1989}, \cite{Grevesse_1998},\cite{Grevesse_2007}, \cite{Asplund_2009}, \cite{Grevesse_2010}, \cite{Scott_2014}, and \cite{Grevesse_2014}.  If the precise method of normalisation is not specified, or if it is done in a non-standard way, it is difficult to convert a set of observations to a different standard.  This note derives the formulae that allow conversion between different solar standards.
  