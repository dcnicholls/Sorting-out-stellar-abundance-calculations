\section{Purpose}

The conventions used by stellar astrophysicists to characterise atomic abundances in stellar spectra appear idiosyncratic to a nebular astrophysicist.  In particular, the normalisation of abundance values to standard solar values depends on the ``standard'' used, and these are numerous. See, for example, \cite{1989GeCoA..53..197A}, \cite{1998SSRv...85..161G},\cite{2007SSRv..130..105G}, \cite{2009ARA&A..47..481A}, \cite{2010Ap&SS.328..179G}, \cite{2015A&A...573A..25S}, \cite{2015A&A...573A..26S} and \cite{2010Ap&SS.328..179G}.  If the precise method of normalisation is not specified, or if it is done in a non-standard way, it is difficult to convert a set of observations to a different standard.  This note derives the formulae that allow conversion between different solar standards.
  
  