\section{Background}

The conventions used by stellar astrophysicists to characterise atomic abundances in stellar spectra differ considerably from those used in nebular astrophysics.  In particular, stellar abundance values are normalised to standard solar values. The resulting values depend on the ``standard'' used, and these are numerous. See, for example, \cite{1989GeCoA..53..197A}, \cite{1998SSRv...85..161G},\cite{2007SSRv..130..105G}, \cite{2009ARA&A..47..481A}, \cite{2010Ap&SS.328..179G}, \cite{2015A&A...573A..25S}, \cite{2015A&A...573A..26S} and \cite{2015A&A...573A..27G}.  If the solar standard used is not specified, or if normalisation is done in a non-standard way, it may be impossible convert the observations to a different standard. Where simple scaling to a specified set of solar abundances is used, it is still a potentially error-prone exercise to convert the data to a different standard. This note derives the formulae that allow conversion between different solar standards.
  
\section{Definitions}
The abundance of oxygen in a stellar spectrum relative to the standard solar abundance is denoted by $[\text{O}]_*$, and the relative abundance of iron by $[\text{Fe}]_*$, where,
\begin{equation}
[\text{O}]_* = \log_{10}\left(\frac{\text{O}}{\text{H}}\right)_* - \log_{10}\left(\frac{\text{O}}{\text{H}}\right)_\odot
\end{equation}
and
\begin{equation}
[\text{Fe}]_* = \log_{10}\left(\frac{\text{Fe}}{\text{H}}\right)_* - \log_{10}\left(\frac{\text{Fe}}{\text{H}}\right)_\odot .
\end{equation}
Here, the suffix ``$_\odot$'' refers to solar abundances and ``$_*$'' to stellar abundances.

\section{Discussion}
Let us denote solar abundance standard 1 with the suffix $_1$ and standard 2 with $_2$. We want to express the normalised abundances in one standard in terms of the normalised abundances in a second standard. Let
\begin{equation}
x_1 = [\text{O}]_{*1}\ ,\  x_2 = [\text{O}]_{*2}\ ,\  y_1 = [\text{Fe}]_{*1}\ ,\  y_2 = [\text{Fe}]_{*2}\ .
\end{equation}
Let us also put
\begin{equation}
a = \log\left(\frac{\text{O}}{\text{H}}\right) ,\  b =  \log\left(\frac{\text{Fe}}{\text{H}}\right)\ .
\end{equation}
and
\begin{equation}
c_1 = \log\left(\frac{\text{O}}{\text{H}}\right)_{\odot 1} ,\  c_2 =  \log\left(\frac{\text{O}}{\text{H}}\right)_{\odot 2} ,\  d_1 = \log\left(\frac{\text{Fe}}{\text{H}}\right)_{\odot 1} ,\  d_2 =  \log\left(\frac{\text{Fe}}{\text{H}}\right)_{\odot 2}

\end{equation}
  
  
  
  
  
  
  
  
  
  
  
  
  
  
  
  
  
  
  
  
  
  
  