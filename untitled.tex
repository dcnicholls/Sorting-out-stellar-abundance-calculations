\section{Background}

The conventions used by stellar astrophysicists to characterise atomic abundances in stellar spectra differ considerably from those used in nebular astrophysics.  In particular, the normalisation of abundance values to standard solar values depends on the ``standard'' used, and these are numerous. See, for example, \cite{1989GeCoA..53..197A}, \cite{1998SSRv...85..161G},\cite{2007SSRv..130..105G}, \cite{2009ARA&A..47..481A}, \cite{2010Ap&SS.328..179G}, \cite{2015A&A...573A..25S}, \cite{2015A&A...573A..26S} and \cite{2015A&A...573A..27G}.  If the precise method of normalisation is not specified, or if it is done in a non-standard way, it may be impossible convert a set of observations to a different standard. Where simple scaling to a specified set of solar abundances is used, it is still a potentially error-prone exercise to convert the data to a different standard. This note derives the formulae that allow conversion between different solar standards.
  
\section{Definitions}
\begin{equation}
[\text{O}]_* = \log_{10}\left(\frac{\text{O}}{\text{H}}\right)_* - \log_{10}\left(\frac{\text{O}}{\text{H}}\right)_\odot
\end{equation}
\begin{equation}
[\text{Fe}]_* = \log_{10}\left(\frac{\text{Fe}}{\text{H}}\right)_* - \log_{10}\left(\frac{\text{Fe}}{\text{H}}\right)_\odot
\end{equation}
  
  
  
  
  
  
  